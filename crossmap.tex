%% BioMed_Central_Tex_Template_v1.06
%%                                      %
%  bmc_article.tex            ver: 1.06 %
%                                       %

%%IMPORTANT: do not delete the first line of this template
%%It must be present to enable the BMC Submission system to
%%recognise this template!!

%%%%%%%%%%%%%%%%%%%%%%%%%%%%%%%%%%%%%%%%%
%%                                     %%
%%  LaTeX template for BioMed Central  %%
%%     journal article submissions     %%
%%                                     %%
%%          <8 June 2012>              %%
%%                                     %%
%%                                     %%
%%%%%%%%%%%%%%%%%%%%%%%%%%%%%%%%%%%%%%%%%


%%%%%%%%%%%%%%%%%%%%%%%%%%%%%%%%%%%%%%%%%%%%%%%%%%%%%%%%%%%%%%%%%%%%%
%%                                                                 %%
%% For instructions on how to fill out this Tex template           %%
%% document please refer to Readme.html and the instructions for   %%
%% authors page on the biomed central website                      %%
%% http://www.biomedcentral.com/info/authors/                      %%
%%                                                                 %%
%% Please do not use \input{...} to include other tex files.       %%
%% Submit your LaTeX manuscript as one .tex document.              %%
%%                                                                 %%
%% All additional figures and files should be attached             %%
%% separately and not embedded in the \TeX\ document itself.       %%
%%                                                                 %%
%% BioMed Central currently use the MikTex distribution of         %%
%% TeX for Windows) of TeX and LaTeX.  This is available from      %%
%% http://www.miktex.org                                           %%
%%                                                                 %%
%%%%%%%%%%%%%%%%%%%%%%%%%%%%%%%%%%%%%%%%%%%%%%%%%%%%%%%%%%%%%%%%%%%%%

%%% additional documentclass options:
%  [doublespacing]
%  [linenumbers]   - put the line numbers on margins

%%% loading packages, author definitions

%\documentclass[twocolumn]{bmcart}
% uncomment this for twocolumn layout and comment line below
\documentclass{bmcart}

%%% Load packages
%\usepackage{amsthm,amsmath}
%\RequirePackage{natbib}
%\RequirePackage{hyperref}
\usepackage[utf8]{inputenc} %unicode support
\usepackage{graphicx}
\usepackage{multirow}
%\usepackage[applemac]{inputenc} %applemac support if unicode package fails
%\usepackage[latin1]{inputenc} %UNIX support if unicode package fails

\usepackage{etoolbox}
\makeatletter
\patchcmd{\@addmarginpar}{\ifodd\c@page}{\ifodd\c@page\@tempcnta\m@ne}{}{}
\makeatother
\reversemarginpar


%%%%%%%%%%%%%%%%%%%%%%%%%%%%%%%%%%%%%%%%%%%%%%%%%
%%                                             %%
%%  If you wish to display your graphics for   %%
%%  your own use using includegraphic or       %%
%%  includegraphics, then comment out the      %%
%%  following two lines of code.               %%
%%  NB: These line *must* be included when     %%
%%  submitting to BMC.                         %%
%%  All figure files must be submitted as      %%
%%  separate graphics through the BMC          %%
%%  submission process, not included in the    %%
%%  submitted article.                         %%
%%                                             %%
%%%%%%%%%%%%%%%%%%%%%%%%%%%%%%%%%%%%%%%%%%%%%%%%%


%\def\includegraphic{}
%\def\includegraphics{}



%%% Put your definitions there:
\startlocaldefs
\endlocaldefs


%%% Begin ...
\begin{document}

%%% Start of article front matter
\begin{frontmatter}

\begin{fmbox}
\dochead{Software}

%%%%%%%%%%%%%%%%%%%%%%%%%%%%%%%%%%%%%%%%%%%%%%
%%                                          %%
%% Enter the title of your article here     %%
%%                                          %%
%%%%%%%%%%%%%%%%%%%%%%%%%%%%%%%%%%%%%%%%%%%%%%

\title{gn-crossmap: A Tool to Verify Scientific Names Checklists}

%%%%%%%%%%%%%%%%%%%%%%%%%%%%%%%%%%%%%%%%%%%%%%
%%                                          %%
%% Enter the authors here                   %%
%%                                          %%
%% Specify information, if available,       %%
%% in the form:                             %%
%%   <key>={<id1>,<id2>}                    %%
%%   <key>=                                 %%
%% Comment or delete the keys which are     %%
%% not used. Repeat \author command as much %%
%% as required.                             %%
%%                                          %%
%%%%%%%%%%%%%%%%%%%%%%%%%%%%%%%%%%%%%%%%%%%%%%

\author[
   addressref={aff1},
   corref={aff1},
   email={mozzheri@illinois.edu}
]{\inits{DYM}\fnm{Dmitry Y.} \snm{Mozzherin}}
\author[
   addressref={aff2},
   email={wouter.addink@naturalis.nl}
]{\inits{WA}\fnm{Wouter} \snm{Addink}}
\author[
   addressref={aff3},
   email={rui.figueira@iict.pt}
]{\inits{RF}\fnm{Rui} \snm{Figueira}}
\author[
   addressref={aff4},
   email={wouter.koch@artsdatabanken.no}
]{\inits{WK}\fnm{Wouter} \snm{Koch}}
\author[
   addressref={aff4},
   email={toril.moen@artsdatabanken.no}
]{\inits{TLM}\fnm{Toril L.} \snm{Moen}}

%%%%%%%%%%%%%%%%%%%%%%%%%%%%%%%%%%%%%%%%%%%%%%
%%                                          %%
%% Enter the authors' addresses here        %%
%%                                          %%
%% Repeat \address commands as much as      %%
%% required.                                %%
%%                                          %%
%%%%%%%%%%%%%%%%%%%%%%%%%%%%%%%%%%%%%%%%%%%%%%

\address[id=aff1]{%                    % unique id
  \orgname{University of Illinois},    % university, etc
  \street{1816 South Oak St.},         %
  \city{Champaign},                    % city
  \state{IL},
  \postcode{61820},
  \cny{US}                             % country
}
\address[id=aff2]{%                    % unique id
  \orgname{Naturalis},
  \street{Postbus 9517, 2300 RA},
  \city{Leiden},                       % city
  \cny{Holland}
}
\address[id=aff3]{%                    % unique id
  \orgname{Instituto de Investigação Científica e Tropical},
  \street{R Junqueira 5,30,86 Lisboa},
  \city{Lisbon},                       % city
  \postcode{1300-342},
  \cny{Portugal}                       % country
}
\address[id=aff4]{%                    % unique id
  \orgname{Norwegian Biodiversity Information Centre},
  \street{7491 Trondheim},
  \cny{Norway}
}

%%%%%%%%%%%%%%%%%%%%%%%%%%%%%%%%%%%%%%%%%%%%%%
%%                                          %%
%% Enter short notes here                   %%
%%                                          %%
%% Short notes will be after addresses      %%
%% on first page.                           %%
%%                                          %%
%%%%%%%%%%%%%%%%%%%%%%%%%%%%%%%%%%%%%%%%%%%%%%

\end{fmbox}% comment this for two column layout

%%%%%%%%%%%%%%%%%%%%%%%%%%%%%%%%%%%%%%%%%%%%%%
%%                                          %%
%% The Abstract begins here                 %%
%%                                          %%
%% Please refer to the Instructions for     %%
%% authors on http://www.biomedcentral.com  %%
%% and include the section headings         %%
%% accordingly for your article type.       %%
%%                                          %%
%%%%%%%%%%%%%%%%%%%%%%%%%%%%%%%%%%%%%%%%%%%%%%

\begin{abstractbox}

\begin{abstract} % abstract
  \parttitle{Background}
  Large checklists that include biological names require that the names are validated if we are to promote interoperability of data sources. Lists may present the same names in different ways or have errors in spelling or author information. Validation allows typographical
  mistakes and erroneous names to be identified as a prelude to replacing incorrect items with names that are endorsed by one or more taxonomic authorities.  Usually, lists of names are stored in a some sort of a spreadsheet, and
  the curator of the list needs to compare each entry with a
  nomenclatural source like Index Fungorum, ZooBank, or IPNI, and/or with a taxonomic sources like AlgaeBase or with a federating portal such as Catalogue of Life. Comparing such lists by hand is tedious
  and the process may generate its own errors.
  %comment Paddy: change verification to validation?
  
  
  \parttitle{Results}
  We created a convenience tool gn-crossmap which intelligently matches
  names from lists and writes the results back into the source. The tool is
  able to process 1 million names in about 10 hours.
  
  
  \parttitle{Conclusions}
  We introduce a tool for biodiversity informaticians -- gnCrossmap, which
  greatly simplifies accelerates mapping of names from a source to one or more authoritative sources of   names.
  

\end{abstract}

%Comment paddy I believe the paper should be set in the broadest relevant context, and this should be evident in both the abstract and Intro.    What is the goal? Is the goal here simply to try to map names in any source to CoL.  Or a tool that can map names in any source to any other list.  The long term goal should be the second. We list all of the possible problems, and then the possible solutions.  There can be problems with incorrect spellings, problems with different legitimate variants of name-strings, use of junior synonyms.  That is, the system should be customizable to suit different users. There are, for example, governmental agencies that are not allowed to change the list of names and they MUST turn back a contemporary name to an outdated name; there is another class of use case that acknowledges that taxonomists hold different views and want to have a framework of names that they can live with. Then there are idiosyncratic cases - such as the Australian Fisheries that require users to adopt common names. Then, with the problems dissected out, we can then consider the strategies.  It seems that there are four options: management by hand, algorithmic normalization, algorithmic standardization and algorithmic resolution.  We should distinguish normalization - mapping to a reference list; and standardization where everyone maps to the same reference list.  CoL has always been a champion of standardization. GN is a champion of algorithmic resolution that is informed by authoritative lists. }



%%%%%%%%%%%%%%%%%%%%%%%%%%%%%%%%%%%%%%%%%%%%%%
%%                                          %%
%% The keywords begin here                  %%
%%                                          %%
%% Put each keyword in separate \kwd{}.     %%
%%                                          %%
%%%%%%%%%%%%%%%%%%%%%%%%%%%%%%%%%%%%%%%%%%%%%%

\begin{keyword}
\kwd{biodiversity}
\kwd{scientific name}
\kwd{matcher}
\end{keyword}

% MSC classifications codes, if any
%\begin{keyword}[class=AMS]
%\kwd[Primary ]{}
%\kwd{}
%\kwd[; secondary ]{}
%\end{keyword}

\end{abstractbox}
%
%\end{fmbox}% uncomment this for twcolumn layout

\end{frontmatter}

%%%%%%%%%%%%%%%%%%%%%%%%%%%%%%%%%%%%%%%%%%%%%%
%%                                          %%
%% The Main Body begins here                %%
%%                                          %%
%% Please refer to the instructions for     %%
%% authors on:                              %%
%% http://www.biomedcentral.com/info/authors%%
%% and include the section headings         %%
%% accordingly for your article type.       %%
%%                                          %%
%% See the Results and Discussion section   %%
%% for details on how to create sub-sections%%
%%                                          %%
%% use \cite{...} to cite references        %%
%%  \cite{koon} and                         %%
%%  \cite{oreg,khar,zvai,xjon,schn,pond}    %%
%%  \nocite{smith,marg,hunn,advi,koha,mouse}%%
%%                                          %%
%%%%%%%%%%%%%%%%%%%%%%%%%%%%%%%%%%%%%%%%%%%%%%

%%%%%%%%%%%%%%%%%%%%%%%%% start of article main body
% <put your article body there>

\section*{Background}

\begin{itemize}
  \item Many reasons people make checklists of names
    \begin{itemize}
      \item Citizen scientists
      \item Taxonomists
      \item Governmental agencies with interests in particular taxa - fisheries, endangered species, invasive species
      \item Organizations: Bird list US, IUCN
      \item Countries create checklists
    \end{itemize}
  \item Example usecases \marginpar{Use cases from Rio's email}
    \begin{itemize}
      \item Norwegian National Checklist
      \item Portuguese Plant List
      \item Preparation of a citizen scientist checklist for publication
        through GBIF to national checklist
      \item Quality check of national GBIF data sets against a reference
        checklist.
    \end{itemize}
  \item When checklists grow they are often in a form that is hard to maintain, for example to ensure that each name is current
  \item Need to check for misspellings, and existence of names
  \item Need to check for outdated names
  \item Created a list that checks
  \item Problems -- hard to check against resources as names are spelled
    differently; for which parser was developed
  \item gn-crossmap created to normalize sources and queries to match them
  \item gn-crossmap matches one resource at a time 6 ways
\end{itemize}

\section*{Implementation}
\begin{itemize}
  \item Implemented as a ``mashup'' to GN resolver
  \item Ruby gem and Ruby web-interface
  \item Installation
    \begin{itemize}
      \item as a gem
      \item as a web-interface
      \item as a docker container
    \end{itemize}
\end{itemize}

\section*{Results and Discussion}
\begin{itemize}
  \item Why it is hard to match a source to a source
    \begin{itemize}
      \item going through web-interfaces of data sources is slow
      \item it is hard to get format right
      \item most often people put data into spreadsheets of their own
        formatting
    \end{itemize}
  \item why it is hard to match name to a name
    \begin{itemize}
      \item differences in spelling
      \item differences in authorship and ranks
      \item some names have homonyms
      \item some names have chresonyms
      \item some names are misspelled
      \item some names do not exist in database
      \item some names do not exist
    \end{itemize}
  \item introduce gn-crossmap
  \item requirements
    \begin{itemize}
      \item able to work with many formats
      \item able to map unknown fields to DarwinCore terms
      \item able to work with datasets of thousands and millions of rows
      \item finds exact matches, canonical form matches, fuzzy matches, partial
        matches
      \item preserves original data
      \item gives information about type of a match (currently used, synonym
        etc)
      \item gives information about confidence score
    \end{itemize}
  \item current problems and future work
    \begin{itemize}
      \item infraspecific ranks in botany
      \item intelligent authorship type recognition
      \item ability to compare concepts
    \end{itemize}
\end{itemize}


\section*{Conclusions}
\begin{itemize}
  \item introduce gn-match
  \item gn-crossmap solves existing problems...
  \item gn-crossmap future work
    \begin{itemize}
      \item infraspecific ranks recognition (botany mostly)
      \item more intelligent work with taxon concepts
      \item more intelligent work with authorships
    \end{itemize}
\end{itemize}

\section*{Availability and Requirements}

Where to find the program etc.

\section*{Abbreviations}


\section*{Author's Contributions}

Who did what

\section*{Acknowledgements}

Grant and people to mention

%%%%%%%%%%%%%%%%%%%%%%%%%%%%%%%%%%%%%%%%%%%%%%%%%%%%%%%%%%%%%
%%                  The Bibliography                       %%
%%                                                         %%
%%  Bmc_mathpys.bst  will be used to                       %%
%%  create a .BBL file for submission.                     %%
%%  After submission of the .TEX file,                     %%
%%  you will be prompted to submit your .BBL file.         %%
%%                                                         %%
%%                                                         %%
%%  Note that the displayed Bibliography will not          %%
%%  necessarily be rendered by Latex exactly as specified  %%
%%  in the online Instructions for Authors.                %%
%%                                                         %%
%%%%%%%%%%%%%%%%%%%%%%%%%%%%%%%%%%%%%%%%%%%%%%%%%%%%%%%%%%%%%

% if your bibliography is in bibtex format, use those commands:
\bibliographystyle{bmc-mathphys} % Style BST file
\bibliography{ref}      % Bibliography file (usually '*.bib' )

% or include bibliography directly:
% \begin{thebibliography}
% \bibitem{b1}
% \end{thebibliography}

\end{document}
